\documentclass[]{article}
\usepackage{amsmath}
\usepackage{graphicx}

\newcommand{\intt}{\ensuremath{\int\limits_0^\beta}\ d\tau}
\DeclareMathOperator{\atan}{atan}
\DeclareMathOperator{\sign}{sign}

% Title Page
\title{Stochastic Optimization Method for Analytic Continuation: implementation notes}
\author{Igor Krivenko}

\begin{document}
\maketitle

This document is a diverse collection of notes, explaining various details of the present SOM implementation.
It is supposed to serve as an addition to the original paper by Andrey S. Mishchenko (see \verb|som.pdf|).

\section{Supported integral kernels}
\subsection{Fermionic thermal Green's function in imaginary time}

\begin{equation}
	G(\tau) = -\int\limits_{-\infty}^\infty
	d\epsilon \frac{e^{-\tau\epsilon}}{1+e^{-\beta\epsilon}} A(\epsilon).
\end{equation}

Integral kernel applied to a rectangle,
\begin{equation}
	\hat K(\tau;\omega)*R(c,w,h) = \left\{ 
	\begin{array}{ll}
	h(\Lambda_0(c + w/2) - \Lambda_0(c - w/2)), & \tau = 0 \\
	h(\Lambda_0(w/2 - c) - \Lambda_0(-w/2-c)), & \tau = \beta \\
	h(\Lambda(\tau;c + w/2) - \Lambda(\tau;c - w/2)), & \mathrm{otherwise}
	\end{array}\right.
\end{equation}
\begin{equation}
	\Lambda_0(\Omega) = \left\{ 
	\begin{array}{ll}
	-\frac{1}{\beta}\log(1 + e^{\beta\Omega}), & \Omega < 0 \\
	-\frac{1}{\beta}\log(1 + e^{-\beta\Omega}) - \Omega, & \Omega \ge 0
	\end{array}\right.
\end{equation}
\begin{equation}
	\Lambda(\tau;\Omega) = \left\{
	\begin{array}{ll}
	\sum_{n=0}^\infty\frac{-\beta}{[\beta(2n+1)-\tau][\beta(2n+2)-\tau]},
	& \Omega = 0 \\
	\sum_{n=0}^\infty\frac{(-1)^{n+1}e^{\Omega(\beta(n+1)-\tau)}}{\beta(n+1)-\tau},
	& \Omega < 0 \\
	-\frac{\pi}{\beta \sin(\pi\tau/\beta)}
	-\sum_{n=0}^\infty\frac{(-1)^{n+1}e^{-\Omega(\beta n +\tau)}}{\beta n+\tau},
	& \Omega > 0 \\
	\end{array}
	\right.
\end{equation}
Infinite sums in the definition of $\Lambda(\tau;\Omega)$ are calculated for each $\tau$-point $G(\tau)$ is given on, and for a regular set of $\Omega$-points. A cubic spline interpolation in the $\Omega$-direction is then constructed for the sake of performance. $\Lambda(\tau;\Omega)$ approaches its known $\Omega\to-\infty$ and $\Omega\to+\infty$ limits exponentially, so that one can use the following interpolation
segment,
\begin{equation}
	\Omega \in [\frac{\log[tolerance\cdot(1-\tau/\beta)]}{\beta-\tau};
				\frac{\log[tolerance\cdot(\tau/\beta)]}{-\tau}].
\end{equation}

\subsection{Fermionic thermal Green's function at imaginary frequencies}
\begin{equation}
	G(i\omega_n) = \int\limits_{-\infty}^\infty
	d\epsilon \frac{1}{i\omega_n-\epsilon} A(\epsilon).
\end{equation}

Integral kernel applied to a rectangle,
\begin{equation}
	\hat K(\tau;\omega)*R(c,w,h) = 
	h \log\left(\frac{i\omega_n - c + w/2}{i\omega_n - c - w/2}\right).
\end{equation}

\subsection{Fermionic thermal Green's function in Legendre polynomial basis}
\begin{equation}
	G(\ell) = -\int\limits_{-\infty}^\infty
	d\epsilon \frac{\beta\sqrt{2\ell+1}(-\mathrm{sgn}(\epsilon))^\ell i_{\ell}(\beta|\epsilon|/2)}
	{2\cosh(\beta\epsilon/2)} A(\epsilon)
\end{equation}
$i_\ell(x)$ is the modified spherical Bessel function of the first kind.

Integral kernel applied to a rectangle,
\begin{equation}
	\hat K(\ell;\omega)*R(c,w,h) = h (\Lambda(\ell;c+w/2) - \Lambda(\ell;c-w/2)) 
\end{equation}
\begin{equation}
	\Lambda(\ell;\Omega) = (-\mathrm{sgn}(\Omega))^{\ell+1}\sqrt{2\ell+1}
	\int_0^{|\Omega|\beta/2} \frac{i_\ell(x)}{\cosh(x)} dx
\end{equation}

The integral $\int_0^z \frac{i_\ell(x)}{\cosh(x)} dx$ is rather inconvenient for fast numerical evaluation. It cannot be expressed in terms of elementary functions or of a quickly converging series. Moreover, it grows logarithmically for $z\to\infty$, which makes spline interpolation infeasible.

These difficulties urge us to use a combined evaluation scheme. The scheme is based on the following expression for the integrand:
\begin{equation}\label{il_cosh_series}
	\frac{i_\ell(x)}{\cosh(x)} =
	\frac{e^x}{e^x+e^{-x}}\sum_{k=0}^\ell(-1)^k
		\frac{a_k(\ell+1/2)}{x^{k+1}} +
	\frac{e^{-x}}{e^x+e^{-x}}\sum_{k=0}^\ell(-1)^{n+1}
		\frac{a_k(\ell+1/2)}{x^{k+1}},
\end{equation}
where $a_k(\ell+1/2)$ are coefficients of the Bessel polynomials,
\begin{equation}
	a_k(\ell+1/2) = \frac{(l+k)!}{2^k k!(l-k)!} =
	\prod_{i=1}^k \frac{1}{2i}[\ell-k+2i-1][\ell-k+2i].
\end{equation}

For large $x$ (high-frequency region) the integrand can be approximated as
\begin{equation}\label{il_cosh_series_high}
	\frac{i_\ell(x)}{\cosh(x)} \approx
		\sum_{k=0}^\ell(-1)^k \frac{a_k(\ell+1/2)}{x^{k+1}}.
\end{equation}

For each $\ell$ the boundary value $x_0$ between the low- and high- frequency regions is found by a direct evaluation of (\ref{il_cosh_series}) and (\ref{il_cosh_series_high}). $x_0\approx15$ for $\ell=0$ (discrepancy on the level of $10^{-13}$) and grows as $\ell^2$ for higher $\ell$.

In the low-energy region we do the integral $F^<(z) \equiv \int_0^z \frac{i_\ell(x)}{\cosh(x)} dx$ on a fixed $z$-mesh, $z\in[0;x_0]$, using the adaptive Simpson's method. Results of the integrations are used to construct a cubic spline interpolation of $F^<(z)$.

In the high-energy region we use the asymptotic form (\ref{il_cosh_series_high}) to do the integral,
\begin{equation}
	F^>(z)|_{z>x_0} = F^<(x_0) +
		\int_{x_0}^z dx \sum_{k=0}^\ell(-1)^k \frac{a_k(\ell+1/2)}{x^{k+1}},
\end{equation}
\begin{equation}
	\int_{x_0}^z dx \sum_{k=0}^\ell(-1)^k \frac{a_k(\ell+1/2)}{x^{k+1}} =
	\left.\left\{
		\log(x) +
		\sum_{k=1}^\ell (-1)^{k+1}\frac{a_k(\ell+1/2)}{x^k k}
	\right\}\right|_{x_0}^z.
\end{equation}


\section{Fit quality criterion for complex functions}

In section 4.1 Mishchenko introduces a special quantity $\kappa$, which characterizes the fit quality of a given particular solution
$A(\omega)$ (eq. 55).
\begin{equation}
	\kappa = \frac{1}{M-1}\sum_{m=2}^M\theta(-\Delta(m)\Delta(m-1)),
\end{equation}
where $\Delta(m) = (\tilde G(m) - G(m))/\mathcal{S}(m)$ is the deviation function. The proposed expression for $\kappa$ makes  apparently no sense for complex $G(m)$ and $\Delta(m)$, e.g. when $G$ is given as a function of Matsubara frequencies.

Our generalization consists in a replacement
\begin{multline}
	\theta(-\Delta(m)\Delta(m-1)) \mapsto
	\frac{1}{2}\left[1 - \frac{\Re[\Delta(m)\Delta^*(m-1)]}{|\Delta(m)\Delta^*(m-1)|} \right]=\\=
	\frac{1}{2}\left[ 1 - \cos[\arg(\Delta(m)) - \arg(\Delta(m-1))] \right].
\end{multline}
According to the modified definition, two adjacent values of $\Delta$ are considered anti-correlated, if the complex phase shift between them exceeds $\pm\pi/2$. In the case of real-valued quantities the phase shift is always either 0, or $\pi$.

\section{Probability density function for the parameter change}

Section 3.4 of \verb|som.pdf| contains a dubious expression for the probability
density function $\mathcal{P}\sim (\max(|\delta\xi_\mathrm{min}|, |\delta\xi_\mathrm{max}|)/|\delta\xi|)^\gamma$, where $\gamma\gg1$. It is
obviously divergent at $\delta\xi\to0$ and cannot be properly normalized, if
ends of $[\delta\xi_\mathrm{min};\delta\xi_\mathrm{max}]$ segment have different signs.

We use a different density function, which is finite everywhere on $\delta\xi\in[\delta\xi_\mathrm{min};\delta\xi_\mathrm{max}]$, and is qualitatively similar to that of Mishchenko.
\begin{equation}
	\mathcal{P}(\delta\xi) = \mathcal{N}
	\exp\left(-\gamma \frac{|\delta\xi|}{L}\right), \quad
	L \equiv \max(|\delta\xi_\mathrm{min}|, |\delta\xi_\mathrm{max}|),
\end{equation}
\begin{equation}
	\mathcal{N} = \frac{\gamma}{L}\left[
		\sign(\delta\xi_\mathrm{min})(e^{-\gamma|\delta\xi_\mathrm{min}|/L} - 1) +
		\sign(\delta\xi_\mathrm{max})(1 - e^{-\gamma|\delta\xi_\mathrm{max}|/L})
	\right]^{-1}.
\end{equation}

\end{document}
